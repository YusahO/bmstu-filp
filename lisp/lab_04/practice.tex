\chapter{Практические задания}

\section{Задание 1}

Чем принципиально отличаются функции \texttt{cons}, \texttt{list}, \texttt{append}?
\begin{enumerate}
	\item \texttt{cons}~--- базисная функция, принимающая 2 аргумента и объединяющая их значения в точечную пару;
	\item \texttt{list}~--- функция, принимающая произвольное количество аргументов и возвращающая список из их значений;
	\item \texttt{append}~--- функция, принимающая произвольное количество аргументов, производящая копирование всех переданных аргументов, кроме последнего, и возвращающая список, содержащий все переданные в качестве аргументов элементы.
\end{enumerate}

\noindent Пусть:
\begin{itemize}[label="", noitemsep]
	\item \texttt{(setf lst1 '(a b c))}
	\item \texttt{(setf lst2 '(d e))}
\end{itemize}
Каковы результаты вычисления следующих выражений?

\includelistingpretty{task_01.lsp}{lisp}{}

\section{Задание 2}

Каковы результаты вычисления следующих выражений и почему?
\begin{enumerate}
	\item \texttt{(reverse '(a b c)) -> (C B A)}
	\\ \texttt{reverse} возвращает копию списка, содержащую элементы исходного списка в обратном порядке;
	
	\item \texttt{(reverse '(a b (c (d)))) -> ((C (D)) B A)}
	\\ \texttt{reverse} работает только со списковыми ячейками верхнего уровня;
	
	\item \texttt{(reverse '(a)) -> (A)}
	
	\item \texttt{(reverse ()) -> NIL}
	
	\item \texttt{(reverse '((a b c))) -> ((A B C))}
	\\ \texttt{reverse} работает только со списковыми ячейками верхнего уровня, \texttt{(A~B~C)}~--- единственный элемент списка на этом уровне;
	
	\item \texttt{(last '(a b c)) -> (C)}
	\\ \texttt{last} возвращает последний элемент списка;
	
	\item \texttt{(last '(a)) -> (A)}
	
	\item \texttt{(last '((a b c))) -> ((A B C))}
	\\ \texttt{last} работает только со списковыми ячейками верхнего уровня, \texttt{(A~B~C)}~--- единственный элемент списка на этом уровне;
	
	\item \texttt{(last '(a b (c))) -> (C)}
	\\ \texttt{last} работает только со списковыми ячейками верхнего уровня, \texttt{(C)}~--- последний элемент списка на этом уровне;
	
	\item \texttt{(last ()) -> NIL}
\end{enumerate}  

\section{Задание 3}

Написать, по крайней мере, два варианта функции, которая возвращает последний элемент своего списка-аргумента.

\includelistingpretty{task_03.lsp}{lisp}{}

\section{Задание 4}

Написать, по крайней мере, два варианта функции, которая возвращает свой список аргумент без последнего элемента.

\includelistingpretty{task_04.lsp}{lisp}{}

\section{Задание 5}

Напишите функцию swap-first-last, которая переставляет в списке-аргументе первый и последний элементы.

\includelistingpretty{task_05.lsp}{lisp}{}

\section{Задание 6}

Написать простой вариант игры в кости, в котором бросается две правильные кости.
Если сумма выпавших очков равна 7 или 11~--- выигрыш, если выпало (1, 1) или (6, 6)~--- игрок имеет право снова бросить кости, во всех остальных случаях ход переходит ко второму игроку, но запоминается сумма выпавших очков. 
Если второй игрок не выигрывает абсолютно, то выигрывает тот игрок, у которого больше очков. 
Результаты игры и значения выпавших костей выводить на экран с помощью \texttt{print}.

\includelistingpretty{task_06.lsp}{lisp}{}

\section{Задание 7}

Написать функцию, которая по своему списку-аргументу \texttt{lst} определяет является ли он палиндромом.

\includelistingpretty{task_07.lsp}{lisp}{}

\section{Задание 8}

Напишите \underline{свои} необходимые функции, которые обрабатывают таблицу из 4-х точечных пар: (страна . столица), и возвращает по стране~--- столицу, а по столице~--- страну.

\includelistingpretty{task_08.lsp}{lisp}{}

\section{Задание 9}
Написать функцию, которая умножает на заданное число-аргумент первый числовой элемент списка из заданного 3-х элементного списка-аргумента, когда
\begin{itemize}
	\item все элементы списка~--- числа;
	\item элементы списка~--- любые объекты.
\end{itemize}

\includelistingpretty{task_09.lsp}{lisp}{}