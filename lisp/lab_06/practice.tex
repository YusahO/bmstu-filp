\chapter{Практические задания}

\section{Задание 1}

Написать хвостовую рекурсивную функцию \texttt{my-reverse}, которая развернет верхний уровень своего списка-аргумента \texttt{lst}.

\includelistingpretty{task_01.lsp}{lisp}{}

\section{Задание 2}

Написать функцию, которая возвращает первый элемент списка-аргумента, который сам является непустым списком.

\includelistingpretty{task_02.lsp}{lisp}{}

\section{Задание 3}

Написать функцию, которая выбирает из заданного списка только те числа, которые больше 1 и меньше 10.

\includelistingpretty{task_03.lsp}{lisp}{}

\section{Задание 4}

Напишите рекурсивную функцию, которая умножает на заданное число-аргумент все числа из заданного списка-аргумента, когда
\begin{enumerate}[label=\alph*)]
	\item все элементы списка~--- числа,
	\item элементы списка~--- любые объекты.
\end{enumerate}

\includelistingpretty{task_04.lsp}{lisp}{}

\section{Задание 5}

Напишите функцию \texttt{select-between}, которая из списка-аргумента, содержащего только числа, выбирает только те, которые расположены между двумя указанными границами-аргументами и возвращает их в виде списка (упорядоченного по возрастанию).

\includelistingpretty{task_05.lsp}{lisp}{}

\section{Задание 6}

Написать рекурсивную версию (с именем \texttt{rec-add}) вычисления суммы чисел заданного списка:
\begin{enumerate}[label=\alph*)]
	\item одноуровневого смешанного,
	\item структурированного.
\end{enumerate}

\includelistingpretty{task_06.lsp}{lisp}{} 

\section{Задание 7}

Написать рекурсивную версию с именем \texttt{recnth} функции \texttt{nth}.

\includelistingpretty{task_07.lsp}{lisp}{} 

\section{Задание 8}

Написать рекурсивную функцию \texttt{allodd}, которая возвращает t, когда все элементы списка нечетные.

\includelistingpretty{task_08.lsp}{lisp}{} 

\section{Задание 9}

Написать рекурсивную функцию, которая возвращает первое нечетное число из списка (структурированного), возможно создавая некоторые вспомогательные функции.

\includelistingpretty{task_09.lsp}{lisp}{}

\section{Задание 10}

Используя cons-дополняемую рекурсию с одним тестом завершения, написать функцию, которая получает как аргумент список чисел, а возвращает список квадратов этих чисел в том же порядке.

\includelistingpretty{task_10.lsp}{lisp}{}
