\chapter{Практические задания}

\section{Задание 1}

Напишите функцию, которая уменьшает на 10 все числа из списка-аргумента этой функции, проходя по верхнему уровню списковых ячеек.

\includelistingpretty{task_01.lsp}{lisp}{}

\section{Задание 2}

Написать функцию, которая получает как аргумент список чисел, а возвращает список квадратов этих чисел в том же порядке.

\includelistingpretty{task_02.lsp}{lisp}{}

\section{Задание 3}

Написать функцию, которая умножает на заданное число-аргумент все числа из заданного списка-аргумента, когда
\begin{enumerate}[label=\alph*)]
	\item все элементы~--- числа,
	\item элементы списка~--- любые объекты.
\end{enumerate}

\includelistingpretty{task_03.lsp}{lisp}{}

\section{Задание 4}

Написать функцию, которая по своему списку-аргументу \texttt{lst} определяет является ди он палиндромом (то есть равны ли \texttt{lst} и \texttt{(reverse lst))}, для одноуровнего смешанного списка.

\includelistingpretty{task_04.lsp}{lisp}{}

\section{Задание 5}

Используя функционалы, написать предикат \texttt{set-equal}, который возвращает t, если два его множества-аргумента содержат одни и те же элементы, порядок которых не имеет значения.

\includelistingpretty{task_05.lsp}{lisp}{}

\section{Задание 6}

Написать функцию, \texttt{select-between}, которая из списка-аргумента, содержащего только числа, выбирает только те, которые расположены между двумя указанными числами~--- границами-аргументами и возвращает их в виде списка (упорядоченного по возрастанию).

\includelistingpretty{task_06.lsp}{lisp}{}

\section{Задание 7}

Написать функцию, вычисляющую декартово произведение двух своих списков-аргументов.

\includelistingpretty{task_07.lsp}{lisp}{}

\section{Задание 8}

Почему так реализовано \texttt{reduce}, в чем причина?
\begin{itemize}[noitemsep, label=""]
	\item \texttt{(reduce \#'+ ()) -> 0}
	\item \texttt{(reduce \#'* ()) -> 1}
\end{itemize}

При передаче пустого списка в функцию \texttt{reduce} она возвращает значение \texttt{initial-value}, которое для сложения равно 0, а для умножения~--- 1.

\section{Задание 9}

Пусть \texttt{list-of-list} список, состоящий из списков. Написать функцию, которая вычисляет сумму длин всех элементов.

\includelistingpretty{task_09.lsp}{lisp}{}