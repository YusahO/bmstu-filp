\chapter{Практическая часть}

\section{Задание}

Создать базу знаний <<Собственники>>, дополнив (и минимально изменив) базу знаний, хранящую знания (лаб. 7), знаниями о дополнительной собственности владельца:
\begin{itemize}
	\item <<Телефонный справочник>>: Фамилия, №тел, Адрес~--- структура (Город, Улица, №дома, №кв),
	\item <<Автомобили>>: Фамилия\_владельца, Марка, Цвет, Стоимость и др.,
	\item <<Вкладчики банков>>: Фамилия, Банк, Счет, Сумма и др.
\end{itemize}

Преобразовать знания об автомобиле к форме знаний о собственности.
Вид собственности (кроме автомобиля):
\begin{itemize}
	\item Строение: стоимость и другие его характеристики;
	\item Участок: стоимость и другие его характеристики;
	\item Водный\_транспорт: стоимость и другие его характеристики.
\end{itemize}

Описать и использовать вариантный домен \textbf{Собственность}.
Владелец может иметь, но только один объект каждого вида собственности (это касается и автомобиля), или не иметь некоторых видов собственности.
Используя конъюнктивное правило и разные формы задания одного вопроса (пояснять, для какого задания~--- какой вопрос), обеспечить возможность поиска:
\begin{enumerate}
	\item названий всех объектов собственности заданного субъекта,
	\item названий и стоимости всех объектов собственности заданного субъекта,
	\item разработать правило, позволяющее найти суммарную стоимость всех объектов собственности заданного субъекта.
\end{enumerate}

Для 2-го пункта и одной фамилии составить таблицу, отражающую конкретный порядок работы системы, с объяснениями порядка работы и особенностей использования доменов (указать конкретные T1 и T2 и полную подстановку на каждом шаге).

\includelistingpretty
	{lab_02.pro}{prolog}{}

\clearpage
\section{Порядок работы системы}

Вопрос: \texttt{property\_price(``Orlov'', PropType, Price).}

	\footnotesize
	\begin{xltabular}{\textwidth}{|l|X|Y|}
		\hline
		\textbf{№ шага} &
		\textbf{Сравниваемые термы: результат; подстановка, если есть} &
		\textbf{Дальнейшие действия: прямой ход или откат} \\ \hline
		
		1 & 
		Сравнение:
		\texttt{property\_price(``Orlov'', PropType, Price) == phonebook(``Ivanov'', 100, make\_address(``Moscow'', ``Taganskaya'', 18, 34))} \newline\newline
		Унификация: неудача &
		Прямой ход, переход к следующему предложению \\ \hline
		
		2 --  & \dots & \dots \\ \hline
		
		 &
		Сравнение:
		\texttt{property\_price(``Orlov'', PropType, Price) == property\_price(Surname, car, Price)} \newline\newline
		Унификация: успех \newline\newline
		Подстановка: \texttt{\{Surname = ``Orlov'', PropType = car, Price = Price\}} & 
		Переход к телу правила. \newline\newline
		Унификация\newline
		\texttt{owner(``Orlov'', car(Price, \_, \_, \_))} \\ \hline
		
		 &
		Сравнение:
		\texttt{owner(``Orlov'', car(Price, \_, \_, \_)) == phonebook(``Ivanov'', 100, make\_address(``Moscow'', ``Taganskaya'', 18, 34))} \newline\newline
		Унификация: неудача &
		Прямой ход, переход к следующему предложению \\ \hline
		
		 --  & \dots & \dots \\ \hline
		
		 &
		Сравнение:
		\texttt{owner(``Orlov'', car(Price, \_, \_, \_)) == owner(``Orlov'', car(5100000, ``Lada'', black, ``P132YC797''))} \newline\newline
		Унификация: успех \newline\newline
		Подстановка: \texttt{\{Price = 5100000\}} & 
		Решение найдено, в качестве побочного эффекта получена подстановка \texttt{\{Property = car, Price = 5100000\}} \newline\newline 
		Откат с целью найти все возможные решения.
		Реконкретизация переменной Price.
		Переход к следующему предложению.\\ \hline
		
		 &
		Сравнение:
		\texttt{owner(``Orlov'', car(Price, \_, \_, \_)) == owner(``Vaselinina'', car(1100000, ``Nissan'', gray, ``K452HH123''))} \newline\newline
		Унификация: неудача &
		Прямой ход, переход к следующему предложению. \\ \hline
		
		 --  & \dots & \dots \\ \hline
		 
		 &
		 &
		Конец базы знаний~--- откат.
		Реконкретизация переменных Surname, Property.
		Переход к следующему предложению относительно шага N.\\ \hline
		
		&
		Сравнение:
		\texttt{property\_price(``Orlov'', PropType, Price) == property\_price(Surname, building, Price)} \newline\newline
		Унификация: успех \newline\newline
		Подстановка: \texttt{\{Surname = ``Orlov'', PropType = building, Price = Price\}} & 
		Переход к телу правила. \newline\newline
		Унификация\newline
		\texttt{owner(``Orlov'', building(Price, \_,))} \\ \hline
		
		 &
		Сравнение:
		\texttt{owner(``Orlov'', car(Price, \_, \_, \_)) == phonebook(``Ivanov'', 100, make\_address(``Moscow'', ``Taganskaya'', 18, 34))} \newline\newline
		Унификация: неудача &
		Прямой ход, переход к следующему предложению \\ \hline
		
		--  & \dots & \dots \\ \hline
		
		&
		Сравнение:
		\texttt{owner(``Orlov'', building(Price, \_)) == owner(``Orlov'', building(1000000, make\_address(``Murmansk'', ``Lenina'', 1, 9))).} \newline\newline
		Унификация: успех \newline\newline
		Подстановка: \texttt{\{Price = 1000000\}} & 
		Решение найдено, в качестве побочного эффекта получена подстановка \texttt{\{Property = building, Price = 1000000\}} \newline\newline 
		Откат с целью найти все возможные решения.
		Реконкретизация переменной Price.
		Переход к следующему предложению.\\ \hline
		
		&
		Сравнение:
		\texttt{owner(``Orlov'', building(Price, \_, \_, \_)) == 	owner(``Vaselinina'', land(300000, 56))} \newline\newline
		Унификация: неудача &
		Прямой ход, переход к следующему предложению \\ \hline
		
		--  & \dots & \dots \\ \hline
		
		 &
		 &
		Конец базы знаний~--- откат.
		Реконкретизация переменных Surname, Property.
		Переход к следующему предложению относительно шага N.\\ \hline
		
		&
		Сравнение:
		\texttt{property\_price(``Orlov'', PropType, Price) == property\_cost(Surname, car, Cost)} \newline\newline
		Унификация: неудача &
		Прямой ход, переход к следующему предложению \\ \hline
				
		--  & \dots & \dots \\ \hline
		
		&
		&
		Конец базы знаний.
		Завершение работы. \newline\newline
		На вопрос удалось получить ответ <<да>>, поэтому в качестве побочного эффекта возвращено 2 подстановки.
		\\ \hline
	\end{xltabular}