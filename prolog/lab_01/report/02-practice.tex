\chapter{Практическая часть}

\section{Задание}

Разработать свою программу~--- <<Телефонный справочник и автомобили>>.
Абоненты могут иметь несколько телефонов.
Протестировать работу программы, используя разные вопросы.

\begin{itemize}
	\item <<Телефонный справочник>>: Фамилия, №тел, Адрес~--- структура (Город, Улица, №дома, №кв),
	\item <<Автомобили>>: Фамилия\_владельца, Марка, Цвет, Стоимость, Номер.
\end{itemize}
Владелец может иметь несколько телефонов, автомобилей.
В разных городах есть однофамильцы, в одном городе фамилия уникальна.

Обеспечить возможность поиска: по Марке и Цвету автомобиля найти Фамилию, Город, Телефон.

\includelistingpretty
	{lab_01.pro}{prolog}{}

\noindent Результат работы данной программы
\includelistingpretty
	{out_01.txt}{prolog}{}

\noindent Результат работы программы со вторым вариантом вопроса:
\includelistingpretty
	{out_02.txt}{prolog}{}

\noindent Результат работы программы с третьим вариантом вопроса:
\includelistingpretty
	{out_03.txt}{prolog}{}
	
\noindent Результат работы программы с четвертым вариантом вопроса:
\includelistingpretty
	{out_04.txt}{prolog}{}
	
\section{Выводы}

Программа <<Телефонный справочник>> на Prolog представляет собой базу знаний и вопрос.
Структура этой программы:
\begin{enumerate}
	\item DOMAINS~--- раздел описания доменов;
	\item PREDICATES~--- раздел описания предикатов;
	\item CLAUSES~--- раздел описания предложений базы знаний;
	\item GOAL~--- раздел описания внутренней цели (вопроса).
\end{enumerate}

Программа реализуется посредством описания базы знаний и последующим заданием вопроса.
Используя базу знаний, система пытается найти такие множества значений переменных, при которых на поставленный вопрос можно ответить <<Да>>.