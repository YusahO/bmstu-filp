\chapter{Практические задания}

\section{Задание 1}

Представить следующие списки в виде списочных ячеек:
\begin{enumerate}
	\item \texttt{`(open close halph)}
	\includeimage
		{task_11}
		{f}
		{H}
		{0.8\textwidth}
		{}
	\item \texttt{`((open1) (close2) (halph3))}
	\includeimage
		{task_12}
		{f}
		{H}
		{0.8\textwidth}
		{}
	\item \texttt{`((one) for all (and (me (for you))))}
	\includeimage
		{task_13}
		{f}
		{H}
		{0.8\textwidth}
		{}
	\item \texttt{`((TOOL) (call))}
	\includeimage
		{task_14}
		{f}
		{H}
		{0.8\textwidth}
		{}
	\item \texttt{`((TOOL1) ((call2)) ((sell)))}
	\includeimage
		{task_15}
		{f}
		{H}
		{0.8\textwidth}
		{}
	\item \texttt{`(((TOOL) (call)) ((sell)))}
	\includeimage
		{task_16}
		{f}
		{H}
		{0.8\textwidth}
		{}
\end{enumerate}

\section{Задание 2}

Используя только функции CAR и CDR, написать выражения, возвращающие элементы заданного списка:
\begin{enumerate}
	\item второй;
	\includelistingpretty
		{task_21.lisp}{lisp}{}
	\item третий;
	\includelistingpretty
		{task_22.lisp}{lisp}{}
	\item четвертый.
	\includelistingpretty
		{task_23.lisp}{lisp}{}
\end{enumerate}

\section{Задание 3}

Что будет в результате вычисления выражений?

\begin{enumerate}
	\item \texttt{(CAADR '((blue cube) (red pyramid)))}
	
	red
	\item \texttt{(CDAR '((abc) (def) (ghi)))}
	
	Nil
	\item \texttt{(CADR '((abc) (def) (ghi)))}
	
	(def)
	\item \texttt{(CADDR '((abc) (def) (ghi)))}
	
	(ghi)
\end{enumerate}

\section{Задание 4}

Напишите результат вычисления выражений и объясните, как он получен:

\begin{table}[H]
	\begin{tabularx}{\textwidth}{|X|X|}
		\hline
		Выражение & Результат вычисления \\ \hline
		\texttt{(list 'Fred 'and 'Wilma)} & (FRED AND WILMA) \\ \hline
		\texttt{(list 'Fred '(and Wilma))} & (FRED (AND WILMA)) \\ \hline
		\texttt{(cons Nil Nil)} & (NIL) \\ \hline
		\texttt{(cons T Nil)} & (T) \\ \hline
		\texttt{(cons Nil T)} & (NIL . T) \\ \hline
		\texttt{(list Nil)} & (NIL) \\ \hline
		\texttt{(cons '(T) Nil)} & ((T)) \\ \hline
		\texttt{(list '(one two) '(free temp))} & ((ONE TWO) (FREE TEMP)) \\ \hline
		\texttt{(cons 'Fred '(and Wilma))} & (FRED AND WILMA) \\ \hline
		\texttt{(cons 'Fred '(Wilma))} & (FRED WILMA) \\ \hline
		\texttt{(list Nil Nil)} & (NIL NIL) \\ \hline
		\texttt{(list T Nil)} & (T NIL) \\ \hline
		\texttt{(list Nil T)} & (NIL T) \\ \hline
		\texttt{(cons T (list Nil))} & (T NIL) \\ \hline
		\texttt{(list '(T) Nil)} & ((T) NIL) \\ \hline
		\texttt{(list '(one two) '(free temp))} & ((ONE TWO) FREE TEMP) \\ \hline
	\end{tabularx}
\end{table}

\section{Задание 5}

Написать лямбда-выражение и соответствующую функцию:

\begin{enumerate}
	\item функция (f ar1 ar2 ar3 ar4), возвращающая ((ar1 ar2) (ar3 ar4));
	\includelistingpretty
		{task_51.lisp}{lisp}{}
	\includeimage
		{task_51}
		{f}
		{H}
		{1\textwidth}
		{}
	\item функция (f ar1 ar2), возвращающая ((ar1)(ar2));
	\includelistingpretty
		{task_52.lisp}{lisp}{}
	\includeimage
		{task_52}
		{f}
		{H}
		{0.8\textwidth}
		{}
	\item функция (f ar1), возвращающая (((ar1))).
	\includelistingpretty
		{task_53.lisp}{lisp}{}
	\includeimage
		{task_53}
		{f}
		{H}
		{0.3\textwidth}
		{}
\end{enumerate}
