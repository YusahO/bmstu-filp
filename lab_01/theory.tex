\chapter{Теоретические вопросы}

\section{Элементы языка: определение, синтаксис, представление в памяти}

Вся информация (данные и программы) в Lisp представляется в виде символьных выражений~--- S-выражений. По определению
\begin{center}
	\texttt{S-выражение ::= <атом> | <точечная пара>}.
\end{center}

Атомы:
\begin{itemize}
	\item символы (идентификаторы)~--- синтаксически~--- набор литер (букв и цифр), начинающихся с буквы;
	\item специальные символы~--- \{T, Nil\} (используются для обозначения логических констант);
	\item самоопределимые атомы~--- натуральные числа, дробные числа (например 2/3), вещественные числа, строки~--- последовательность символов, заключенных в двойные апострофы (например ``abc'');
\end{itemize}

Более сложные данные~--- списки и точечные пары (структуры) строятся из унифицированных структур~--- блоков памяти~--- бинарных узлов.

Определения:

\begin{flushleft}
	\texttt{Точечные пары ::= (<атом>.<атом>) | (<атом>.<точечная пара>) | (<точечная пара>.<атом>) | (<точечная пара>.<точечная пара>};
\end{flushleft}

\begin{flushleft}
	\texttt{Список ::= <пустой список> | <непустой список>}, 
	
	где \texttt{<пустой список> ::= () | Nil},
		
	\hspace{6.8mm} \texttt{<непустой список> ::= (<первый элемент>.<хвост>)},
		
	\hspace{6.8mm} \texttt{<первый элемент> ::= <S-выражение>},
		
	\hspace{6.8mm} \texttt{<хвост> ::= <список>}.
\end{flushleft}

Синтаксически любая структура (точечная пара или список) заключается в круглые скобки; \texttt{( A.B )}~--- точечная пара, \texttt{( A )}~--- список из одного элемента; пустой список изображается как \texttt{Nil} или \texttt{()}.

\texttt{( A.( B.( C.( D () ))))}, допустимо изображение списка последовательностью атомов, разделенных пробелами~--- \texttt{( A B C D )}.

Элементы списка могут, в свою очередь, быть списками (любой список заключается в круглые скобки), например: \texttt{( A ( B C ) ( D (E) ))}.
Таким образом, синтаксически наличие скобок является признаком структуры~--- списка или точечной пары.

Любая непустая структура Lisp в памяти представляется списковой ячейкой, хранящей два указателя: на голову (первый элемент) и хвост~--- все остальное.

\section{Особенности языка Lisp. Структура программы. Символ апостроф.}

Особенности языка Lisp:
\begin{itemize}
	\item бестиповый язык;
	\item символьная обработка данных;
	\item любая программа может интерпретироваться как функция с одним или несколькими аргументами;
	\item автоматизированное динамическое распределение памяти, которая выделяется блоками;
	\item программа может быть представлена как данные, то есть программа может изменять саму себя.
\end{itemize}

Символ апостроф~--- сокращенное обозначение функции \texttt{quote}, блокирующей вычисление своего аргумента.

\section{Базис языка Lisp. Ядро языка.}

Базис языка~--- минимальный набор обозначений, в которые можно свести все правильные (то есть вычислимые) формулы системы.

Базис Lisp образуют:
\begin{itemize}
	\item атомы;
	\item структуры;
	\item базовые функции;
	\item базовые функционалы.
\end{itemize}

Ядро языка~--- базис + наиболее употребимые функции языка.